\documentclass{article}

\usepackage{arxiv}

\usepackage[utf8]{inputenc} % allow utf-8 input
\usepackage[T1]{fontenc}    % use 8-bit T1 fonts
\usepackage{hyperref}       % hyperlinks
\usepackage{url}            % simple URL typesetting
\usepackage{booktabs}       % professional-quality tables
\usepackage{amsfonts}       % blackboard math symbols
\usepackage{nicefrac}       % compact symbols for 1/2, etc.
\usepackage{microtype}      % microtypography
\usepackage{lipsum}		% Can be removed after putting your text content
\usepackage{graphicx}
\usepackage{wrapfig}
\usepackage{natbib}
\usepackage{doi}
\usepackage{amsmath}
\usepackage{amssymb}
\usepackage{verbatim}
\usepackage[english,russian]{babel}

\title{Метрический анализ пространства параметров глубоких нейросетей.}

%\date{September 9, 1985}	% Here you can change the date presented in the paper title
%\date{} 					% Or removing it

\author{Эрнест Р.~Насыров\thanks{Use footnote for providing further
		information about author (webpage, alternative
		address)} \\
	ФПМИ\\
	МФТИ\\
	Долгопрудный\\
	\texttt{nasyrov.rr@phystech.edu} \\
	%% examples of more authors
    }
	
	%% \AND
	%% Coauthor \\
	%% Affiliation \\
	%% Address \\
	%% \texttt{email} \\
	%% \And
	%% Coauthor \\
	%% Affiliation \\
	%% Address \\
	%% \texttt{email} \\
	%% \And
	%% Coauthor \\
	%% Affiliation \\
	%% Address \\
	%% \texttt{email} \\


% Uncomment to remove the date
%\date{}

% Uncomment to override  the `A preprint' in the header
%\renewcommand{\headeright}{Technical Report}
%\renewcommand{\undertitle}{Technical Report}
\renewcommand{\shorttitle}{\textit{arXiv} Template}

%%% Add PDF metadata to help others organize their library
%%% Once the PDF is generated, you can check the metadata with
%%% $ pdfinfo template.pdf
\hypersetup{
pdftitle={Метрический анализ пространства параметров глубоких нейросетей},
pdfsubject={q-bio.NC, q-bio.QM},
pdfauthor={Эрнест Р.~Насыров},
pdfkeywords={нет слова},
}

\begin{document}
\maketitle

\begin{abstract}
В работе исследуется проблема снижения размерности пространства параметров модели. В ее рамках решается задача восстановления временного ряда. В качестве модели восстановления ряда используются различные автоенкодеры. В работе проводится метрический анализ пространства параметров автоенкодера. Новизна заключается в том, что отдельные параметры модели - случайные величины - собираются в векторы – многомерные случайные величины, анализ взаимного расположения которых в пространстве и представляет предмет исследования нашей работы. Этот анализ позволит снизить количество параметров модели, сделать выводы о значимости параметров, произвести их отбор. Для определения положения вектора параметров в пространстве оцениваются его матожидание и матрица ковариации с помощью методов bootstrap и variational inference. Эксперименты проводятся на моделях SSA, RNN и VAE на задачах предсказания синтетических временных рядов, квазипериодических показаний акселерометра, периодических видеоданных.
\end{abstract}


% keywords can be removed
%\keywords{First keyword \and Second keyword \and More}


\section{Introduction}

% 1. История + связанные понятия (выбор оптимально модели ...).

С развитием технологий скорость обработки данных растет, они становятся более сложными, большей размерности. Такие высокоразмерные данные часто избыточны, хотя и содержат много полезной информации, что представляет сложность для их эффективной обработки и использования. Отсюда возникает задача снижения размерности признакового описания объекта, которая привлекла большое внимание ученых. Ее базовый принцип состоит в том, чтобы отобразить высокоразмерное признаковое пространство в низкоразмерное, сохраняя важную информацию о данных \citep{jia2022feature}.

На текущий момент известно много методов снижения размерности данных. В работе \citep{ornek2019nonlinear} снижения размерности достигается за счет построения дифференцируемой функции эмбеддинга в низкоразмерное представление, а в \citep{cunningham2014dimensionality} обсуждаюся линейные методы. 

В работе \citep{isachenko2022quadratic} задача снижения размерности решается для предсказания движения конечностей человека по электрокортикограмме с использованием  метода QPFS, учитывающем мультикоррелированность и входных, и целевых признаков.



Наряду с задачей снижения размерности входных данных стоит задача выбора оптимальной структуры модели. В случае оптимизации структуры нейросети, большое внимание уделено изучению признакового пространства модели. В работах \citep{hassibi1993optimal} и \citep{dong2017learning} применяется метод OBS (Optimal Brain Surgeon), состоящий в удалении весов сети с сохранением ее качества аппроксимации, причем выбор удаляемых весов производится с помощью вычисления гессиана функции ошибки по весам.

В статье \citep{грабовой2019определение} приводится метод первого порядка, решающий задачу удаления весов, основанный на нахождении дисперсии градиента функции ошибки по параметру и анализе ковариационной матрицы параметров, а в статье \citep{грабовой2020введение} нерелевантные веса не удаляют, а прекращают их обучение.


% 2. А вот тут пошла актуальность
Приведенные выше задачи понижения размерности данных и выбора оптимальной структуры нейросети основаны на исследовании пространства входных данных и пространства признаков соответственно. На взгляд авторов статьи, существенный недостаткок предыдущих работ состоит в том, что в них анализируются \textit{отдельные} параметры (скаляры) моделей и их взаимозависимость. Тем самым не учитывается, что на входные данные действуют \textit{вектора} параметров посредством скалярных произведений, то есть упускается из виду простая \textit{структура} преобразования.

В данной работе мы решаем задачу восстановления временного ряда, в рамках которой занимаемся проблемой снижения пространства параметров модели, основанном на анализе сопряженного пространства ко входному, которое связывает входное пространство и пространство параметров. 

Наше исследование в большой степени полагается на простоту устройства глубоких нейросетей, которые являются композицией линейных и простых нелинейных функций (функций активации). Составной блок нейросети может быть описан формулой  $y=\sigma(Wx), y \in \mathbb{R}^m, x \in \mathbb{R}^n, W \in \mathbb{R}^{m \times n}б \sigma: \mathbb{R} \to \mathbb{R}$. Тогда если раньше элементы $W_{ij}$ исследовались по-отдельности, как скаляры, то в нашей работе изучаются векторы-строки $w_1, \dots, w_m: W = \begin{pmatrix}
w_1^T\\
\dots\\
w_m^T\\
\end{pmatrix}$. В нейросети эти строки обычно называются \textit{нейронами}. В SSA $\sigma = Id$, а матрица $W=W_k$ это приближение истинной матрицы фазовых траекторий $X$ (матрицы Ганкеля) суммой $k$ элементарных матриц.

Мы предполагаем, что каждая точка фазовой траектории распределена нормально вокруг своего матожидания. Тогда обучающая выборка $\mathcal{D}=(\boldsymbol{X}, \boldsymbol{y})$ это набор случайных величин, поэтому и результат обучения модели на ней, то есть веса модели тоже будут случайными. 

В работе исследуется положение случайных векторов параметров модели $w_i$ в метрическом пространстве. С помощью методов bootstrap и Variational Inference \citep{hastie2009elements} оцениваются их матожидания $e_i=Ew_i$ и ковариационные матрицы $var(w_i) = \boldsymbol{A}^{-1}_i$. Мы работаем в гипотезе, что эти векторы $w_i$ распределены нормально, таким образом пара $(e_i, \boldsymbol{A}^{-1}_i)$ полностью описывает вероятностное распределение вектора $w_i$. 

В качестве графического анализа пространства производится изображение положения этих векторов как смеси гауссианов \ref{ris:gauss_mixture}, а также изображение $95\%$ доверительной области каждого вектора \ref{ris:gauss_conf_area}.

% Картинки, иллюстрирующие сказанное выше
\begin{wrapfigure}{r}{0.30\textwidth}

\includegraphics[width=0.25\textwidth]{gaussian_mixture.jpg}
\caption{Пример смеси гауссианов для 2-х мерных векторов.}
\label{ris:gauss_mixture}

\includegraphics[width=0.25\textwidth]{gaussian_conf_area.jpg}
\caption{Пример доверительных областей для 3-х мерных векторов.}
\label{ris:gauss_conf_area}

\end{wrapfigure}


Уменьшение размерности достигается за счет метрического анализа пространства векторов-параметров путем отбора релевантных строк (с малой дисперсией), замены мультикоррелирующих строк на их линейную композицию с помощью обобщения алгоритма QPFS, изучения структуры сообществ строк.

В качестве базовых моделей используются SSA (\citep{golyandina2001analysis}), нелинейный PCA???(А как назвать?, ссылка на источник!), RNN (\citep{bronstein2021geometric}) и NeuroODE (\citep{chen2018neural}).

Задача восстановления временного ряда решается на синтетических данных зашумленного $sin$, данных показания акселерометра в датасете MotionSense3 \citep{malekzadeh2018protecting} (НЕТ ДОСТУПА К ЭТИМ ДАННЫМ: ДОСТУПНЫ ТОЛЬКО ПО ЗАПРОСУ), периодичных видеоданных (НАЙТИ ИХ). 


% А дальше немного про SSA, RNN и VAE (И посмотреть на модели Стрижова!!!)
\newpage

\bibliographystyle{unsrtnat}
\bibliography{references}  %%% Uncomment this line and comment out the ``thebibliography'' section below to use the external .bib file (using bibtex) .





\end{document}
